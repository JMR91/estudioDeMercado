\hrulefill
\subsubsection{Alta de estacionamiento}
	Los datos para dar de alta un estacionamiento son:
	\begin{itemize}
		\item Nombre del estacionamiento.
		\item Ubicaci�n(texto y coordenadas geogr�ficas)
		\item Logo representativo (Opcional)
		\item Horario de servicio  
		\item Niveles y cajones
		\item Cajones para personas con capacidades diferentes
	\end{itemize}
\hrulefill
\subsubsection{Reservaci�n}
	Para que el usuario final pueda reservar un caj�n es necesario que este dado de alta.   

\hrulefill
\subsubsection{Servicios adicionales}
	Se podr�n gestiosnar servicios adicionales(lavado, aspirado, pulido, Etc.). Estos podr�n tener un cargo extra(si as� lo desea el administrador).

\hrulefill
\subsubsection{N�mero de accesos}
	Los estacionamientos podr�n tener de uno a muchos accesos. Estos accesos es donde se registran entradas y salidas.

\hrulefill
\subsubsection{Alta veh�culo}
	Para registrar un veh�culo se solocitaran los siguientes datos:
	\begin{itemize}
			\item Placa
			\item Marca
			\item Modelo				
			\item Color
			\item Detalles(opcional)				
			\item Fotos(opcional)
		\end{itemize}

\hrulefill
\subsubsection{Alta usuario App}
	Los usuarios se podr�n registrar desde la app m�vil. Los datos solicitados son:
	\begin{itemize}
		\item Nombre completo
		\item Tel�fono
		\item M�vil
		\item e-mail
		\item Direcci�n 
	\end{itemize}	

\hrulefill
\subsubsection{Control de pensionados}
	Se podr� agregar multiples pensiones a un cajon siempre y cuando no se traslape los horarios de uso. En horarios donde no este asignado un pensionado ese cajon estara libre para su uso.

\hrulefill
\subsubsection{Control de pensionados}
	Se podr� agregar multiples pensiones a un cajon siempre y cuando no se traslape los horarios de uso. \\Se podr� usar un caj�n siempre y cuando no este en horario de pensi�n. En caso de que entre en uso el horario de pensi�n y no el cajon este ocupado por otro veh�culo se tendr� que reservar un cajon povisional. 
	
\hrulefill
\subsubsection{Alta Pensionado}
	Para dar de alta un pensionada se deber� dar los siguentes datos.
	\begin{itemize}
		\item Nombre completo
		\item Tel�fono
		\item M�vil
		\item e-mail
		\item Direcci�n
		\item Placa del vehiculo registrado
	\end{itemize}
	Otra opcion ser� vincular con un usuario registrado.

\hrulefill	
\subsubsection{Registro valet}
	Al recibir un veh�culo el operador podr� indicar el estado como se resive el veh�culo(con rayaduras, golpes, fata de alguna pieza, Etc.).
	
\hrulefill
\subsubsection{Horario de pensi�n}
	Para dar de alta un pensionado, el usuario final debera estar dado de alta en el sistema como pensionado.
	\begin{itemize}
		\item Placas del veh�culo 
		\item Horario de uso
		\item Vegencia de uso			
	\end{itemize}

\hrulefill
\subsubsection{Tiempo de vida de la informaci�n}
	La informaci�n ser� alamcenada en el sistema, pero no se� eliminada.

\hrulefill
\subsubsection{Historial}
	El cajero solo podra visualizar el hisotrial de accesos y su respectiva informacion en turno.
 
		
