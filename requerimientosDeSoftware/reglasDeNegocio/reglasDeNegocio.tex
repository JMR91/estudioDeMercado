\subsection{RNF}
	\paragraph{Alta de stacionamiento}
		Los datos para dar de alta un estacionamiento son:
		\begin{itemize}
			\item Nombre del estacionamiento.
			\item Ubicaci�n(texto y coordenadas geogr�ficas)
			\item Logo representativo (Opcional)
			\item Horario de servicio  
			\item N�mero de niveles y cajones por nivel
			\item Cajones para personas con capacidades diferentes
		\end{itemize}

\subsection{RNF}
	\paragraph{Reservaci�n}
		Para que el usuario final pueda reservar un caj�n, tendra que ingresar el n�mero de placa del veh�culo.   

\subsection{RNF}
	\paragraph{SaaS}
		El sistema trabajar� bajo una plataforma Cloud, y ser� accedido como un servicio API.

\subsection{RNF}
	\paragraph{Configuraci�n de estacionamiento}
		Un estacionameito podr� operar con las siguientes configuraciones.
		\begin{itemize}
			\item Estacionamiento 
			\item Pensi�n
			\item Valet
			\item Estacionamiento con Pensi�n 					
		\end{itemize}

\subsection{RNF}
	\paragraph{N�mero de accesos}
		Los estacionamientos tpodr�n tener de uno a muchos accesos. Estos acceos es donde se registrarn entradas y salidas.

\subsection{RNF}
	\paragraph{Horario de pensi�n}
		Al registrar un pensionado se deber� indicar.
		\begin{itemize}
			\item Placas del veh�culo 
			\item Horario de uso
			\item Vegencia de uso			
		\end{itemize}
		
\subsection{RNF}
	\paragraph{Tiempo de vida de la informaci�n}
		La informaci�n ser� alamcenada en el sistema, pero no se�a eliminada.
		
		